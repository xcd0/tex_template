% UTF-8 LF
% プリアンブル{{{
\expandafter\ifx\csname ifDividing\endcsname\relax
	\documentclass[12pt,a4j]{jarticle}
	\usepackage{tabularx}
	\usepackage[dvipdfmx]{graphicx}
	\usepackage{enumerate}
	\usepackage{url}
	\usepackage{here}
	\usepackage{graphicx}
	%\usepackage{siunitx}
	\usepackage{multirow}
	\usepackage{listings}
	\usepackage{cases}
	\usepackage{ulem}
	\usepackage{otf}
	\usepackage{listings,jlisting}
	\lstset{
		language={C},
		numbers=left,
		breaklines=true,
		basicstyle={\scriptsize},
		%identifierstyle={\small},
		identifierstyle={\scriptsize},
		%commentstyle={\small\itshape},
		commentstyle={\scriptsize\ttfamily},
		%keywordstyle={\small\bfseries},
		keywordstyle={\scriptsize\ttfamily},
		%ndkeywordstyle={\small},
		ndkeywordstyle={\scriptsize},
		%stringstyle={\small\ttfamily},
		stringstyle={\scriptsize\ttfamily},
		frame={tb},
		columns=[l]{fullflexible},
		xrightmargin=0zw,
		xleftmargin=3zw,
		%numberstyle={\scriptsize},
		numberstyle={\tiny},
		stepnumber=1,
		numbersep=1zw,
		lineskip=-0.5ex
	}
	\graphicspath{{./img/}}
	\begin{document}
\fi
% }}}
%============================================================
% ここに文章を記述する
\section{1/sub.tex}
1/sub.tex





%============================================================
% 数式、画像、表、ソースコード添付の例{{{
\if0
============================================================================

==== 数式の例 ====

\begin{eqnarray}
	&	=	&	\\
	&	=	&	\nonumber \\
	&	=	&	
\end{eqnarray}


==== 画像の例 ====

\begin{figure}[H]
	\centering
	% \includegraphics[width=9cm]{./img/c13.png}
	\caption{あああああ}
	\label{aaaaaa}
\end{figure}


==== 表の例 ====

表のセルを結合したい場合\multirow{}{}{} \multicolumn{}{}{}を使います.
\begin{table}[H]の[H]は大文字です.ちゃんと意味があるので小文字にしないでください.
\scalebox{}{}は表の縦横の幅を拡縮することができます.

\begin{table}[H]
	\centering
	\caption{定電圧源等価回路における回路の内部抵抗}
	\label{c16}
	%\scalebox{1.2}[1.2]{
	\begin{tabular}{|c|c|c|c|c|c|c|} \hline
		&	$P\ [\Omega]$	&	$Q\ [\Omega]$	&	$R\ [\Omega]$	&	$X\ [\Omega]$	&	$R_0\ [\Omega]$ \\ \hline
		&	\multirow{2}{*}{10}	&	\multirow{2}{*}{100}	&	1006.7	&	99.93	&	$1.653\times10^5$ \\ \cline{4-6}
		&	&	&	992	&	99.93	&	$1.635\times10^5$ \\ \hline
		&	\multirow{3}{*}{100}	&	\multirow{3}{*}{1000}	&	1017	&	100.92	&	$1.589\times10^5$ \\ \cline{4-6}
		&	&	&	1013.1	&	100.92	&	$3.162\times10^5$ \\ \cline{4-6}
		&	&	&	1001.4	&	100.92	&	$1.568\times10^5$ \\ \hline
		&	\multirow{4}{*}{1000}	&	\multirow{4}{*}{10000}	&	1050.4	&	100.99	&	$3.166\times10^4$ \\ \cline{4-6}
		&	&	&	1030.2	&	100.99	&	$6.195\times10^4$	\\ \cline{4-6}
		&	&	&	990	&	100.99	&	$6.074\times10^4$	\\ \cline{4-6}
		&	&	&	971	&	100.99	&	$3.048\times10^4$	\\ \hline
	平均値	&	-	&	-	&	-	&	-	&	$1.748\times10^5$	\\ \hline
	\end{tabular}
	%}
\end{table}

==== ソースコードを載せる例 ====

\lstinputlisting[basicstyle=\ttfamily\footnotesize, frame=single]{./ああああああ.txt}

============================================================================
\fi

\expandafter\ifx\csname ifDividing\endcsname\relax
	\end{document}
\fi
% }}}

